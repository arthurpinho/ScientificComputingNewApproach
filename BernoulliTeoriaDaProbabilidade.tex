\documentclass{article}
\title{Bernoulli e o início da teoria da probabilidade}
\begin{document}
\maketitle
	Jacques Bernoulli foi um matemático suíço que nasceu no ano de 1654, e viveu até 1705. Ele teve uma grande importância no avanço da matemática, fazendo estudos em cálculo infinitesimal, geometria analítica e no cálculo exponencial.
		
	Porém, uma de suas maiores contribuições foi no desenvolvimento da teoria de probabilidade. Em 1713, oito anos depois de sua morte, Bernoulli teve o seu trabalho Ars Conjectandi publicado. Este trabalho é dividido em quatro partes: foi abordado o tratado De ludo aleae na primeira parte, uma abordagem breve sobre probabilidade feita por Christiaan Huygens. Prosseguindo, tratou sobre teoria geral de combinções e de permutações, provando pela primeira vez o teorema binomial para potências inteiras positivas. Por fim, nas duas últimas partes, o matemático suíço tratou sobre problemas que ilustram a teoria da probabilidade. Este trabalho é de grande importância, uma vez que ele é considerado o primeiro livro que tratou de probabilidade de uma maneira vasta

\end{document}


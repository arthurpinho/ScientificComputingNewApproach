\documentclass{article}
\title{Estudo sobre influência de atividade física na melhora da depressão e de dependência química}
\begin{document}
\maketitle
	Em sua pesquisa, Eliane Jany Barbanti estudou como pessoas com depressão e pessoas dependentes químicas respondiam a atividades físicas quanto a saúde mental, melhoria geral, aspecto social e vitalidade. Usando a distribuição binomial para medir a melhora de cada pessoa que participou do estudo, ela conseguiu mostrar o efeito positivo de atividades físicas para essas pessoas. Pessoas que realizaram atividades tiveram uma melhora tanto na saúde física, quanto na mental. Essa melhora foi verificada em um estudo inicial com 2 meses de atividade, e depois com 4 meses de atividade.



BARBANTI,E. J.,Efeito da atividade física na qualidade de vida de pacientes com depressão e dependência química, Revista Brasileira de Atividade Física e Saúde(itálico), Florianópolis, v.11, n.1, set. 2012
\end{document}
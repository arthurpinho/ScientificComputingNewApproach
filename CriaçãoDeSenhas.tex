\documentclass{article}
\title{Como criar senhas mais seguras?}
\begin{document}
\maketitle
	Ao criarmos uma nova conta no e-mail, ou em algum site, nos deparamos com a dúvida de como criar uma senha que seja, ao mesmo tempo, segura e fácil de decorar. Para responder isso, Leonardo Carneiro de Araújo, João Pedro Hallack Sansão e Hani Camille Yehia fizeram um estudo sobre como escrever uma senha de uma forma que a dúvida sobre cada caracter seja a maior possível. 
	
	O estudo foi feito observando-se a influência da lei de Zipf na quantidade de dúvida sobre cada caracter de uma senha. A lei de Zipf é uma lei observada na natureza, que, especificamente aplicada em línguagem humana, diz que a frequência de cada palavra em um texto segue a fórmula:
	
	\setlength\parindent{128pt}				$f = 1/n^{s}$
	
	\noindent ,ou seja, em uma lista ordenada das palavras que mas aparecem em um texto,a frequência de uma dada palavra na posição de número n será de um sobre n elevado a s. O expoente s é obtido a partir de dados experimentais.
	
	Ao aplicar a lei de Zipf nesse estudo, chegou-se a conclusão que, para maximizar a entropia, ou seja, maximizar a quantidade de dúvida sobre cada caracter de uma senha, a melhor estratégia é criar o acrônimo de uma frase. Portanto, uma senha difícil de ser quebrada e que é razoavelmente fácil de memorizar seria: eiealdzesrcacdsms . Essa senha foi criada a partir da frase 'é importante estudar a lei de zipf e sua relacao com a criacao de senhas mais seguras', selecionando a primeira letra de cada palavra e concatenando-as.
\end{document}